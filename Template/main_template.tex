\documentclass[letterpaper, onecolumn,10pt]{IEEEtran}

\usepackage{graphicx}
\usepackage{amssymb}
\usepackage{amsmath}
\usepackage{amsthm}

\usepackage{alltt}
\usepackage{float}
\usepackage{color}
\usepackage{url}
\usepackage{listings}
\usepackage{ifthen}


\usepackage[TABBOTCAP, tight]{}

\usepackage{geometry}
\geometry{textheight=8.5in, textwidth=6in}

%random comment

\newcommand{\cred}[1]{{\color{red}#1}}
\newcommand{\cblue}[1]{{\color{blue}#1}}

\usepackage{hyperref}
\usepackage{geometry}
\usepackage{caption}
\usepackage{url}
\usepackage{natbib}

\begin{document}
    \begin{titlepage}
    \newcommand{\HRule}{\rule{\linewidth}{0.5mm}}
    \center
    \textsc{\Large Oregon State University}\\[1.5cm]
    \textsc{\Large ST 314}\\[0.5cm]
    \textsc{\Large Winter 2019}\\[0.5cm]
    \HRule \\[0.4cm]
    { \huge \bfseries Data Analysis One}\\[0.4cm] % Title of your document
    \HRule \\[1.5cm]
    \begin{minipage}{0.4\textwidth}
        \begin{flushleft} \large
        \emph{Author:}\\
        Thomas Noelcke
        \end{flushleft}
    \end{minipage}
    \begin{minipage}{0.4\textwidth}
        \begin{flushright} \large
        \emph{Instructor:} \\
        Katie Jager\\
        \end{flushright}
    \end{minipage}\\[2cm]
		\end{titlepage}
        
        \iffalse
            Part 1. (8 points) For each random variable:
            (1 point) State whether the random variable should be modeled with a Binomial or a Poisson distribution.
            (1 point) Explain your reasoning.
            (1 point) State the parameter values that describe the distribution.
            (1 point) Give the specific probability mass function. 
        \fi
		\section{Part 1}
		
		\iffalse
Part 2: (10 points) Wheel of Fortune is a popular game show on Television.
Contestants spin a wheel and try to guess a correct letter from a word puzzle. If
they guess correctly, they earn the dollar amount from the wheel. If they spin
“bankrupt” or “lose a turn” they get nothing and can’t play. To the right is an
example of the wheel. Watch this video to see an example of someone spinning
the wheel. https://www.youtube.com/watch?v=\_Pv33JWBdY8
The outcome of a spin on the wheel is a discrete random variable. Consider X
the dollar amount spun on the wheel, where Bankrupt and Lose a Turn = \$0, and
Free Play = \$500. There are 24 wedges on the wheel. 

		 a. (2 points) Fill in the probability mass function p(x). The first few are provided as a guide.
b. (2 points) Calculate the Cumulative Density Function F(x). The first few are provided as a guide.
c. (3 points) What is the average dollar amount? Show work!
d. (2 point) What is the most likely dollar amount?
e. (1 point) What is the chance someone spins more than \$600?
		\fi
		\section{Part 2}
		
		\iffalse
Part 3. (6 points) Suppose a contestant spins the wheel three times.
a. (2 points) If spinning a “$0” is considered the event of interest, and each spin is independent of the other
spins, what common discrete distribution best models X the number of “$0” outcomes among 3 spins?
b. (2 points) How likely is it they spin $0 each time? Show work!
c. (2 points) How likely is it they spin $0 at least one time? Show work!
		\fi
		
		\section{part 3}
			
		
		\iffalse
		    Part 4: (6 points) The probabilities in Part 2 are based on probability theory aka “math”. Do these probabilities
stand up when a contestant actually spins the wheel? Go back to the Data Analysis #1 instructions page.
Download the R script titled: Wheel\_of\_Fortune\_Spin\_Script.R , open the file it will automatically open in R.
You need R software on the computer to open the script window. Follow the instructions in the code then
answer the following:
a. (1 point) What value did you spin?
b. (1 point) What is the average of the 1000 simulated spins? How different is this from the average you
calculated in part 2?
c. (1 point) Paste the probability mass function and the plot of the probability mass function from R.
d. (1 point) How different are the simulated probabilities to the theoretical probabilities in part 2?
e. (1 point) Based on the plot is the most likely outcome the same as it is in part 2?
f. (1 point) In general, what action will make the simulated values more like the theoretical ones? 
		\fi
		\section{part 4}
		
		
		\section{Bibliography}
		\bibliography{References}
		\end{document}